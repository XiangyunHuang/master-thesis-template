% 设置英文字体为 Times New Roman
% Windows
% \usepackage[T1]{fontenc}
% \usepackage{mathptmx}
% Linux
\usepackage[OT1]{fontenc}
\usepackage{mathptmx}

\usepackage{booktabs}

% 定义一般页面的页眉和页脚
\usepackage{fancyhdr}
\pagestyle{fancy}
\fancyhf{}
\renewcommand{\headrule}{\hrule width\headwidth \vspace{1.5pt}\hrule width\headwidth}
\fancyhead[EC]{\kaishu 中国矿业大学~(北京) 硕士学位论文}
\fancyhead[OC]{\kaishu \leftmark}
\fancyfoot[C]{\thepage}

\ctexset{
  chapter/name = {,},
  chapter/number = \sf{\arabic{chapter}},
  % section/name = {\S},
  % section/number = \arabic{section}
  section/format += \raggedright,
  contentsname = {目\quad 录},
  abstractname = {摘\quad 要},
  % bibname      = {参~考~文~献}
}

\usepackage{pdfpages}

% 定义所有的图片文件在 figures 子目录下
\graphicspath{{figures/}}

% 处理代码块中英文混合问题
\RecustomVerbatimEnvironment{Highlighting}{Verbatim}{commandchars=\\\{\},formatcom=\xeCJKVerbAddon} 