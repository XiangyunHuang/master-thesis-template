% 设置英文字体 Times New Roman 的等价字体
% Windows
% \usepackage[T1]{fontenc}
% \usepackage{mathptmx}
% Linux
\usepackage[OT1]{fontenc}
\usepackage{mathptmx}

% 为了使用 Arial 字体,用于章节前的数字
\renewcommand{\sfdefault}{phv}

\usepackage{booktabs}

% 章节标题数字样式
\ctexset{
  chapter/name = {,},
  chapter/number = \arabic{chapter},
  chapter/numberformat = \sf, % 数字字体 Arial
  chapter/beforeskip = 12pt, % 标题前后的间距
  chapter/afterskip = 18pt,
  chapter/fixskip = true, % 标题与正文的距离
  chapter/format += \sf\zihao{3},  % 一级标题
  section/numberformat = \rm,    
  section/format += \sf\zihao{4}\raggedright, % 二级标题
  subsection/numberformat = \rm,   
  subsection/format += \sf\zihao{-4}\raggedright,  % 三级标题
  contentsname = {目\quad 录},
  % bibname      = {参\quad 考\quad 文\quad 献},
}

\setlength\parindent{40pt} % 段首缩进40点 即两个汉字

% 定义一般页面的页眉和页脚
\usepackage{fancyhdr}
\pagestyle{fancy}
\fancyhf{}
% 设置文武线
\renewcommand{\headrule}{\hrule height1pt width\headwidth \vspace{1.5pt}\hrule width\headwidth}
% 设置左页页眉
\fancyhead[EC]{\kaishu 中国矿业大学~(北京) 硕士学位论文} % 左页也是奇数页 
% 设置右页页眉
\fancyhead[OC]{\kaishu \leftmark} % 右页也是偶数页
% 设置页脚
\fancyfoot[C]{\thepage} % 没有 E或O 则表示左页和右页一样的设置

% 专为 book 类设置新章节的首页
\fancypagestyle{plain}{ \fancyhf{} %
\fancyhead[EC]{\kaishu 中国矿业大学~(北京) 硕士学位论文}
\fancyhead[OC]{\kaishu \leftmark}
\fancyfoot[C]{\thepage}}

% 用于插入PDF文件
\usepackage{pdfpages}

% 设定图片路径
\graphicspath{{figures/}}
