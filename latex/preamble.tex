% Times New Roman
\setmainfont{Nimbus Roman No9 L}

%% 数学字体
\usepackage{amsmath}
\usepackage{amsfonts} % \mathbb \mathfrak
\usepackage{mathrsfs} % \mathscr

%% 章节标题数字样式
\ctexset{
  chapter/name = {,},
  chapter/number = \arabic{chapter},
  chapter/numberformat = \sf, % 数字字体 Arial
  chapter/beforeskip = 12pt, % 标题前后的间距
  chapter/afterskip = 18pt,
  chapter/fixskip = true, % 标题与正文的距离
  chapter/format += \sf\zihao{3},  % 一级标题
  section/numberformat = \rm,    
  section/format += \sf\zihao{4}\raggedright, % 二级标题
  subsection/numberformat = \rm,   
  subsection/format += \sf\zihao{-4}\raggedright,  % 三级标题
  % contentsname = {目\quad 录},
}

%% 段首缩进40点 即两个汉字
\setlength\parindent{40pt}

\usepackage{fancyhdr}
\pagestyle{fancy}
\fancyhf{}
% 设置文武线
\renewcommand{\headrule}{\hrule height1pt width\headwidth \vspace{3.0pt}\hrule width\headwidth}
% 设置左页页眉
\fancyhead[EC]{\kaishu 中国矿业大学~(北京) 硕士学位论文} % 左页也是奇数页 
% 设置右页页眉
\fancyhead[OC]{\kaishu \leftmark} % 右页也是偶数页
% 设置页脚
\fancyfoot[C]{\thepage} % 没有 E或O 则表示左页和右页一样的设置

% 专为 book 类设置新章节的首页
\fancypagestyle{plain}{ \fancyhf{} %
\fancyhead[EC]{\kaishu 中国矿业大学~(北京) 硕士学位论文}
\fancyhead[OC]{\kaishu \leftmark}
\fancyfoot[C]{\thepage}}

\graphicspath{{figures/}}

\frontmatter